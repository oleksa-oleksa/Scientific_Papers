% !TeX spellcheck = en_GB 
%%%%%%%%%%%%%%%%%%%%%%%%%%%%%%%%%%%%%%%%%%%%%%%%%%%%%%%%%%%%%%%%%%%%%%%%%%%%%%%%
%2345678901234567890123456789012345678901234567890123456789012345678901234567890
%        1         2         3         4         5         6         7         8

\documentclass[letterpaper, 10 pt, conference]{ieeeconf}  % Comment this line out if you need a4paper

%\documentclass[a4paper, 10pt, conference]{ieeeconf}      % Use this line for a4 paper

\IEEEoverridecommandlockouts                              % This command is only needed if 
                                                          % you want to use the \thanks command

\overrideIEEEmargins                                      % Needed to meet printer requirements.

% See the \addtolength command later in the file to balance the column lengths
% on the last page of the document

% The following packages can be found on http:\\www.ctan.org
%\usepackage{graphics} % for pdf, bitmapped graphics files
%\usepackage{epsfig} % for postscript graphics files
%\usepackage{mathptmx} % assumes new font selection scheme installed
%\usepackage{times} % assumes new font selection scheme installed
%\usepackage{amsmath} % assumes amsmath package installed
%\usepackage{amssymb}  % assumes amsmath package installed
\usepackage{url}



\title{\LARGE \bf
Home Digital Voice Assistants: use cases and vulnerabilities
}


\author{\textbf{Oleksandra Baga}\\ 
\textit{\small Master Computer Science, Freie Universit\"at Berlin}\\ 
\textit{\small Sommersemester 2021, Seminar Technische Informatik}\\
{\small oleksandra.baga@gmail.com}}

\begin{document}


\maketitle

\thispagestyle{empty}
\pagestyle{empty}


%%%%%%%%%%%%%%%%%%%%%%%%%%%%%%%%%%%%%%%%%%%%%%%%%%%%%%%%%%%%%%%%%%%%%%%%%%%%%%%%
\begin{abstract}

Smart speakers with voice assistants achieved last years impressive results in speech recognition enabling more seamless interactions between user and a machine but also raise privacy concerns due to their continuously listening microphones. A better understanding of these aspects can help future smart speaker users to make a right decision about the digitalisation of their homes. For these purposes this paper contains as well a result of research about the functionality of digital voice assistance and the real use cases how people are tending to use a device as the research of actual security and privacy concerns including attack surfaces and vulnerabilities.  
\end{abstract}


%% STRUCTURE
\section{INTRODUCTION}
The development of the Deep Learning algorithms and Internet of Things last years is opening up a new era in the use of the digital tools that surround us. A combination of various algorithms in Machine Learning, Deep Learning, speech synthesis, and Natural Language Processing (NLP) to providing services to the users makes it possible to achieve impressive results in speech recognition enabling more seamless interactions between user and a machine. It is realistic now to say that in the coming years digital voice assistances probably will come into the use of every household. They  could  become  embedded  in  users'  day-to-day  routines,  particularly  people  in  a  dependent situation, whether elderly or disabled. 

However, these undeniable advances should not obscure the questions that voice assistants raise from a data protection  perspective,  in  particular  from  the  point  of  view of transparency in the way their system functions \cite{c8}. In the survey by Lau et al. \cite{c5} smart speaker users and non-users were interviewed to find out their arguments for and against adopting this new technology and their privacy perceptions and concerns. Many non-users believe that these devices are not useful at all and companies are not to be trusted. On the other hand, smart speaker users have fewer privacy concerns and rely on companies to safeguard their personal data which think are not interesting to others \cite{c2}. The goal of this research is reality check of privacy concerns and myths circulating about voice assistants and the abilities that they are assumed to have. This paper presents the closer look at digital voice assistance functions for a clearer understanding of the logic behind these systems and the security questions they raise for their users. 

%%%%%%%%%%%

\section{DIGITAL VOICE ASSISTANCES}
\subsection{What is Voice Assistance?}
Over the last years a very significant progress in the development of digital voice assistance is made with various factors contributing to this: improved methods, a significant increase in computing capacity and greater volumes of data available. This is enabling voice assistants in millions of homes today. In this connection a new report from Juniper Research has found that consumers will interact with voice assistants on over 8.4 billion devices by 2024; overtaking the world's population and growing 113\% compared to the 4.2 billion devices expected to be in use by year end 2020 \footnote{https://www.juniperresearch.com/press/number-of-voice-assistant-devices-in-use}. 

Even if someone as an author of this paper is a non-user whatever the reason is, like privacy concerns or confidence that the device is useless, the understanding of what is digital voice assistance is and how it works is important regarding the rising amount of devices. It could be easy imagined that a best friend who oft invites to have a dinner at his place has bought a digital voice assistance or got it as a present on Christmas and even forgot to tell that there is a constantly listening device in their home. 

What is actually a digital voice assistance? A voice assistant is a complex system consisting of several modules to perform different tasks. From a hardware side there are embodied speaker with microphones and some computing capabilities (more or less developed depending on the case). As smart speaker is relatively simple by design and small be size most of the computing and artificial intelligence processing happens in the cloud and not in the device itself \cite{c2}. Because data has to be sent back and forth to centralized data centers a user has to make a request first through the voice-activated device, and then, the voice request gets streamed through the cloud, and here voice gets converted into text \cite{c2}. From a software side there are many modern algorithms used on the backend for processing the user request and implementing  human-machine interaction as such and which includes built-in modules for automatic speech recognition, natural language comprehension and generation,  dialogue and speech synthesis. As was mentioned above in many cases it is done remotely using cloud-based architecture.
After the request processing on the backend a text response will be generated. Finally, the text response goes through the cloud and gets transformed into voice using speaker of the physical digital device and streamed back to the user. After that the voice assistant returns to standby and is constantly listening again to hear a specific wake word uttered by a user. 


\subsection{How does Voice Assistance work?}
\subsection{Third party features}
\subsubsection{Skills and Actions}


%%%%%%%%%%%

\section{USE CASES}
\subsection{Usage by children}
\subsection{title}



%%%%%%%%%%%

\section{VULNERABILITIES}
\subsection{Security and Privacy Concerns}
\subsection{Accidental Recordings}
\subsection{Voice-based remote attacks}
\subsubsection{Voice squatting attack (VSA)}
\subsubsection{Voice Masquerading Attack (VMA)}

%%%
\section{CONCLUSIONS}

A conclusion section is not required. Although a conclusion may review the main points of the paper, do not replicate the abstract as the conclusion. 


%%%%%%%%%%%%%%%%%%%%%%%%%%%%%%%%%%%%%%%%%%%%%%%%%%%%%%%%%%%%%%%%%%%%%%%%%%%%%%%%
\section{COPY AND PASTE CORE IDEAS}
\subsection{About}
\textbf{VPA on IoT devices}. Amazon and Google are two major players in the market of smart speakers with voice-controlled personal assistant capabilities. Since the debut of the first Amazon Echo in 2015, Amazon has now taken 76\% of the U.S. market with an estimate of 15-million devices sold in the U.S. alone in 2017. A unique property of these four devices is that they all forgo conventional I/O interfaces, such as the touchscreen, and also have fewer buttons (to adjust volume or mute), which serves to offer the user a hands-free experience. In another word, one is supposed to command the device mostly by speaking to it. For this purpose, the device is equipped with a microphone circular array designed for 360-degree audio pickup and other technologies like beamforming that enable far-field voice recognition.

Behind these smart devices is a virtual personal assistant, called Alexa for Amazon and Google Assistant for Google, engages users through a two-way conversation. Unlike those serving a smartphone (Siri, for example) that can be activated by a button push, the VPAs for these IoT devices are started with a wake-word like ``Alexa'' or ``Hey Google''. These assistants have a range of capabilities, from weather report, timer setting, to-do list maintenance to voice shopping, hands-free messaging and calling. The user can manage these capabilities through a companion app running on her smartphone.

\subsection{Skills and actions}
Both Amazon and Google enrich the VPAs' capabilities by introducing voice assistant function called skill by Amazon or action by Google. Skills are essentially third-party apps, like those running on smartphones, offering a variety of services the VPA itself does not provide. Examples include Amex, Hands- Free Calling, Nest Thermostat and Walmart. These skills can be conveniently developed with the supports from Amazon and Google, using Alexa Skills Kit [32] and Actions on Google. Indeed, we found that up to November 2017, Alexa already has 23,758 skills and Google Assistant has 1,001. HOW MANY DO WE HAVE NOW?!

Both Amazon Alexa and Google Assistant run a skill market that can be accessed from their companion app on smartphones or web browser for users to discover new skills.

Skills can be started either explicitly or implicitly. Explicit invocation takes place when a user requires a skill by its name from a VPA: for example, saying ?Alexa, talk to Amex? to Alexa triggers the Amex skill for making a payment or checking bank account balances. Such a type of skills is also called custom skills on Alexa.

Implicit invocation occurs when a user tells the voice assistant to perform some tasks without directly calling to a skill name. For example, ?Hey Google, will it rain tomorrow?? will invoke the Weather skill to respond with a weather forecast. Google Assistant identifies and activates a skill implicitly whenever the conversation with the user is under the context deemed appropriate for the skill. This invocation mode is also supported by the Alexa for specific types of skills.

Specifically, to invoke a skill explicitly, the user is expected to use a wake-word, a trigger phrase, and the skill?s invocation name. For example, for the spoken sentence ?Hey Google, talk to personal chef?, ?Hey Google? is the wake-word, ?talk to? is the trigger phrase, and ?personal chef? is the skill invocation name. Note that skill invocation name could be different from skill name, which is intended to make it simpler and easier for users to pronounce. For example, ?The Dog Feeder? has invocation name as the dog; ?Scryb? has invocation name as scribe. When a user invokes a VPA device with its wake-word, the device captures her voice command and sends it to the VPA service provider?s cloud for processing. The cloud performs speech recognition to translate the voice record into text, finds out the skill to be invoked, and then delivers the text, together with the timestamp, device status, and other meta-data, as a request to the skill?s web service. Note that the skill will only receive requests in text format rather than the users? voice recordings.To publish a skill, the developer needs to submit the informa- tion about her skill like name, invocation name, description and the endpoint where the skill is hosted for a certification process. This process aims at ensuring that the skill is functional and meets the VPA provider?s security requirements and policy guidelines.

\subsection{Adversary Model}
today anyone can publish her skill through Amazon and Apple markets, given that these markets have only minimum protection in place to regulate the functions submitted: almost nothing on Amazon before our attacks were reported2, and only the basic check is performed on Google to find duplicated invocation names. once a malicious skill is published, it can be transparently launched by the victim through her voice commands, without being downloaded and installed on her device. Therefore, they can easily affect a large number of VPA IoT devices. 

 we found that for Amazon, such names are not unique skill identifiers: multiple skills with same invocation names are on the Amazon market. Also, skills may have similar or related names.66 different Alexa skills are called cat facts, 5 called cat fact and 11 whose invocation names contain the string ?cat fact?, e.g. fun cat facts, funny cat facts. When such a common name is spoken, Alexa chooses one of the skills based on some undisclosed policies (possibly random as observed in our research). When a different but similar name is called, however, longest string match is used to find the skill. For example, ?Tell me funny cat facts? will trigger funny cat facts rather than cat facts. This problem is less serious for Google, which does not allow duplicated invocation names. However, it also cannot handle similar names

\subsection{Voice-based remote attacks.}
In our research, we analyzed the most popular VPA IoT systems - Alexa and Google Assistant, focusing on the third-party skills deployed to these devices. It is completely feasible for an adversary to remotely attack the users of these popular systems, collecting their private information through their conversations with the systems \cite{c1}. 

\textbf{Voice squatting attack (VSA):} the adversary exploits how a skill is invoked (by a voice command), and the variations in the ways the command is spoken (e.g., phonetic differences caused by accent, courteous expression, etc.) to cause a VPA system to trigger a malicious skill instead of the one the user intends \cite{c1}. For example, one may say ``Alexa, open Capital One please'', which normally opens the skill Capital One, but can trigger a malicious skill Capital One Please once it is uploaded to the skill market. In response to the commands, a malicious skill can pretend to yield control to another skill (switch) or the service (terminate), yet continue to operate stealthily to impersonate these targets and get sensitive information from the user.

More specifically, we first surveyed 156 Amazon Echo and Google Home users and found that most of them tend to use natural languages with diverse expressions to interact with the devices: e.g., ``play some sleep sounds'' . These expressions allow the adversary to mislead the service and launch a wrong skill in response to the user's voice command, such as \textit{some sleep sounds} instead of \textit{sleep sounds} \cite{c1}.

Our further analysis of both Alexa and Google Assistant demonstrates that indeed these systems identify the skill to invoke by looking for the longest string matched from a voice command. From these responses, we found that 50\% of the Amazon Echo users used ?please? at least once in their invocation examples, so did 41\% of the Google Home users.
Also, 28\% users reported that they did open unintended skills when talking to their devices. \cite{c1}.

found that an adversary can intentionally induce confusion by using the name or similar one of a target skill, to trick the user into invoking an attack skill when trying to open the target. For example, the adversary who aims at Capital One could register a skill Capital Won, Capitol One, or Captain One. All such names when spoken by the user could become less distinguishable, particularly in the presence of noise, due to the limitations of today?s speech recognition techniques.

To study voice squatting, we randomly sampled 100 skills each from Alexa and Google assistant markets.  For this purpose, we studied two types of the attacks: voice squatting in which an attack skill carries a phonetically similar invocation name to that of its target skill, and word squatting where the attack invocation name includes the target?s name and some strategically selected additional words (e.g., ?cat facts please?). During this study, we found that a mispronounced invocation name would also trigger the right skill if their pronunciation is close and there is no other registered skills using the mispronounced invocation name. ...we then register ?captain one? as the attack skill?s invocation name, play the original invocation utterance... Such skills were invoked five times each in the test modes of Alexa and Google Assistant.

To study word squatting, we randomly sampled ten skills from each skill markets as the attack targets. For each skill, we built four new skills whose invocation names include the target?s name together with the terms identified from our survey study (Section III-A): for example, ?cat facts please? and ?my cat facts?. On Alexa, an attack skill with the extended name (that is, the target skill?s invocation name together with terms ?please?, ?app?, ?my? and ?the?) was almost always launched by the voice commands involving these terms and the target names. On Google Assistant, however, only the utterance with word ?app? succeeded in triggering the corresponding attack skill, which demonstrates that Google Assistant is more robust against such an attack. However, when we replaced ?my? with ?mai? and ?please? with ?plese?, all such attack skills were successfully invoked by the commands for their target skills (see Table IV). This indicates that the protection Google puts in place (filtering out those with suspicious terms) can be easily circumvented.

\textbf{Voice Masquerading Attack (VMA):}

Google Assistant seems to have protection in place against the impersonation. Specifically, it signals the launch of a skill by speaking ?Sure, here is?, together with the skill name and a special earcon, and skill termination with another earcon. Both Alexa and Google Assistant support voluntary skill termination, allowing a skill to terminate itself right after making a voice response to the user. according to our survey study, 78\% of the participants rely on the response of the skill (e.g. ?Goodbye? or silence) to determine whether a skill has been terminated. This allows an attack skill to fake its termination by providing ?Goodbye? or silent audio in its response while keeping the session alive. 

When sending back a response, both Alexa and Google Assistant let a skill include a reprompt (text content or an audio file), which is played when the VPA does not receive any voice command from the user within a period of time. If the user continues to keep quiet, after another 6 seconds for Alexa and one additional reprompt from Google and follow-up 8-second waiting, the running skill will be forcefully terminated by the VPA. On the other hand, we found in our research that as long as the user says something (even not meant for the skill) during that period, the skill is allowed to send another response together with a reprompt. Adding a silent audio file (up to 90 seconds for Alexa and 120 seconds for Google Assistant) will make it to be able to continue to run at least 102 seconds on Alexa and 264 seconds on Google. This running time can be further extended considering the attack skill attaching the silent audio right after its last voice response to the user (e.g., ?Goodbye?), which gives it 192 seconds on Alexa and 384 on Google Assistant), and indefinitely whenever Alexa or Google Assistant picks up some sound made by the user. In this case, the skill can reply with the silent audio and in the meantime, record whatever it hears.

Only ?exit? is processed by the VPA service and used to forcefully terminate the skill. Through survey study, we found that 91\% of Alexa users used ?stop? to terminate a skill, 36\% chose ?cancel?, and only 14\% opted for ?exit?, which suggests that the user perception is not aligned with the way Alexa works and therefore leaves the door open for the VMA. Consequently, all the information stealing and Phishing attacks caused by the VSA can also happen here. 

THIS WAS SOURCE 1 \cite{c1}\\\\
%%===================

SOURCE 2 \cite{c2}
CHILDREN

Interesting findings came from Sciuto et al. (2018), where authors explored how households incorporate conversational agents into their lives. Specifically, authors ana-lyzed the logs of 75 Alexa users, for a total of 278,654 voice commands. Participants who have owned an Alexa device for at least six months, answered survey questions related to their household use of Alexa. Of the 75 participants, 26 reported having children although data from the log files did not provide any insights into which household member gave each command. Parents that were interviewed, positively recalled their children successfully interacting with Alexa even before interacting with smartphones and other technology devices.

Beirl et  al. (2019), conducted a research about the home usage of Alexa, in a period of three weeks. Six families with children in the age group of 2?13 years were recruited. Results showed that there was much enthu-siasm about how they had interacted with Alexa and how it became part of their family rituals. The interactions with Alexa often resulted in much shared laughter and there were also several instances of teasing. There was also a lot of encouragement, specifi-cally when a more competent family member helped a younger member interact with Alexa. The study concluded that all of the above interactions, contributed to social and emotional bonding, leading to further family cohesion.

Children behavior is investigated by Druga et  al. (2017) \cite{c3} where 26 participants (3-10 years old) interacted with 4 voice assistants, Amazon Alexa, Google Home, Cozmo, and Julie Chatbot. Children enjoyed interaction with voice assistants, while older children perceived their intelligence and thought they could learn from them. The main issue of the interaction with children was getting the assistants to understand their questions although with the help of facilitators and parents, children altered their strategy and became fluent in voice interaction. 

Children behavior is investigated by Druga et  al. (2017) \cite{c3} where 26 participants (3?10 years old) interacted with 4 voice assistants, Amazon Alexa, Google Home, Cozmo, and Julie Chatbot. None of the children expressed suspicion or inquired about how the system worked. After reviewing the logs and audio recordings of all the participants, authors came to the conclusion, that children preferred personified interfaces rather than non-personified and that age played an important role in children?s performance. Older children could get the answer that they needed using less help from provided hints. Since the interaction required children to reformulate questions, most of them needed hints to complete the task. 

Children aged 5 to 6 and their parents? interactions with a smart speaker were also studied by Lovato et al. (2019). Analysis of the results showed that 89\% of children?s ques-tions were transcribed correctly, although only 50\% of children?s questions received a full answer. Children and their parents reported that the provided answers were long or required interpretation. Most children?s questions were about the world around them and they believed that the device is a source of information. 

%%%%%

Security and Privacy Concerns \cite{c2}: 

Privacy is also considered by Hoy (2018) since anyone with access to a voice-activated device can ask it questions, gather information about the accounts and services associated with the device, and ask it to perform tasks. As stated by Horn (2018), since Voice Assistants and Smart Speakers can distinct children?s voices, their specific learning needs are certain to raise questions with the Children?s Online Privacy Protection Act (COPPA) and the Family Educational Rights and Privacy Act (FERPA).

In a survey by Malkin et al. (2019) \cite{c4} sample group was approximately gender-balanced,with 44.0\% self-identifying as female, and the median reported age was 34. Households of 2 or more accounted for 83.6\% of all participants, with a median household size of 3. Regarding 116 smart speaker owners? beliefs, attitudes, and concerns about the recordings that are made and shared by their devices (Amazon and Google), it was found that 56\% of them did not know that their recordings were being permanently stored and that they could review them. It was observed, that while participants did not consider their own recordings important or sensitive, they were more protective of others? recordings, such as children and guests, and were strongly opposed to the use of their data by third parties. Only 3\% of the participants review their recordings and deleted them. Additionally, only 5\% of the participants used the mute button on their device while only 4\% unplugged their device in order to stop listening. The survey concludes that privacy controls are underutilized\cite{c2}


%%%%
Regarding RQ1, early findings from a small number of studies, show that adults (Purington et    al., 2017; Sciuto et    al., 2018; McLean and Osei-Frimpong, 2019; Rzepka, 2019; Song, 2019; Bunyard, 2019) use voice assistants for entertainment purposes, seek-) use voice assistants for entertainment purposes, seek-ing information, making purchases and listening to music. Adults also enjoy interaction and find voice assistants easy to use. Studies involving children (Sciuto et al.,  2018; Beirl et al., 2019; Druga et al., 2017; Yuan et al., 2019; Lovato et al., 2019) conclude that children can interact with smart speakers and voice assistants, making basic requests.\cite{c2}

%%%%%
To detect when a user makes a request, a smart speaker?s multiple microphones continuously listen for the device?sactivation keyword (e.g., ?Alexa? or ?Hey Google?). The smart speaker responds to a request through virtual or physical actions and audio feedback \cite{c5}. Current smart speakers are equipped with some privacy features. While the device?s microphonesare always listening, speech recognition is performed locally by the device until the activationkeyword has been detected, at which point the subsequent voice command is forwarded to themaker?s servers for processing. In addition, most smart speakers are equipped with a physicalbutton to mute the microphones. Companion mobile apps and websites enable users to reviewand delete prior voice interactions with the device should they feel uncomfortable or not wantcompanies to keep particular voice recordings on their servers.\cite{c2}

user participants included mostly primary users \cite{c2}? people who have set up their smart speakerthemselves and connected it to their own accounts. Users mostly placed their smart speakers incentral locations in their homes to maximize utility. Since some rooms in people?s homes are more private than others, we investigated if users considered this when placing their speakers. Smart speakers were commonly placed in central locations in participants? homes, sometimes on dedicated tables situated at intersection points of multiple rooms. In study made by Lau et al. (2018a) \cite{c5} participants reported various uses for their speakers, which were also reflected in their diary responses (frequencies based on diary responses): music playback (14),checking the weather (13), probing smart speaker capabilities (11), controlling other devices (8),asking knowledge questions (6), setting timers (6), checking the time (4), checking the news (3),setting alarms (3), setting reminders (3), checking their calendar (2), purchasing items online (2),playing games (1), and listening to the radio (1). Some utilized multiple different features of their smart speakers, but many stated they used their smart speakers almost exclusively to play music. Seven users controlled other IoT devices in their home with the smart speaker, such as lights,thermostats, and home security systems. This required users to purchase other, potentially expensive,IoT devices that can be linked with smart speakers \cite{c2}

%%%%%%

To understand how people use voice assistants \cite{c6}, we conducted interviews with 19 participantsto explore how voice assistant users made sense of these new technologies. We then collectedAmazon Alexa and Google Home ?histories,? automatically generated logs of commands, to analyzepatterns of use, ultimately analyzing 82 logs totaling 193,665 commands for Amazon Alexa, and88 logs totaling 65,499 commands for Google Home. These logs were categorized into severalmain command categories. Our surveys and data collection mechanisms were approved by ourorganizations? review processes.

\textbf{Music, media and volume:} Based on the log analysis, playing music was the most common use of Amazon Alexa (at 28.5\%) andthe second most used command category for Google Home (at 26.1\%). Users played music basedon genre (e.g., classical music), album (e.g., ?The Fame? by Lady Gaga), or artist (e.g., The Beatles).Voice assistant users also employed Spotify, Pandora, and other music streaming services whenlistening to music. \cite{c6}

Figure 2a shows the heatmap for the music and search on Amazon Alexa aggregated over the24-hour time line. We present the weight of the specific command category as a portion of all other commands throughout that period of time. For Amazon Alexa, the music command was used most heavily between 6 and 10 pm, whilepeaking between 6 and 8 pm. Figure 2b shows a the equivalent heat map for Google Home music and search commands over the 24-hour time line. Similar to the Amazon Alexa heatmap, we find that music was used most heavily between 6 pm and 9 pm. This might arise because users are listening to music while preparing meals at the end of the workday. Around 4.9\% of Amazon Alexa interactions and 5.9\% of Google Home commands were volume related.\cite{c6}

Interviewees did not limit their voice assistant use to music. Some interviewees indicated that they used their voice assistant to access other media. For example, one used Google Home, along with Google Chromecast o operate their Netflix account. \cite{c6}


\textbf{Search}: Search or informational queries was the most prevalent use of Google Home (at 26\%) and secondmost prevalent use for Amazon Alexa (at 19.4\%). The frequency of search command use was highestfor both Amazon Alexa and Google Home was between 5 and 7 pm followed by the time between8 am and noon.As Tables 4 and 5 show, one of the most popular terms was ?song? for both Amazon Alexaand Google Home. Users used the search command to ask questions about music they listened to,specifically the name of a song they are listening to, or the name of the artist singing a particular song, etc. \cite{c6}

Some respondents emphasized the use of the search feature when interacting with family and friends. -- random questions, like trivia questions, or like some facts,  sports scores  or check stockmarket value.. lot for cooking uses it for converting measurements, you know howmany teaspoons are in a cup\cite{c6}

\textbf{Timers}. We can also see that the use of timer command category in both Google Home and Amazon Alexa logs is between 5 and 7 pm. This corresponds to the time users might be cooking dinner at theend of the workday.  This corresponds to the time users might be cooking dinner at theend of the workday. For example, Gavin notes that they use timers mostly for cooking purposes. Timers could also be used to set reminders for users. Table 5 shows an example where a usersets a timer to ?remind me to make a smoothie at 11 am.? \cite{c6}

Most interviewees also noted that they used VAs as Alarms. The terms used in VA logs aspresented in Tables 4 and 5 show the Alarm category includes words like ?set? as in set the Alarmand ?snooze? when snoozing the alarm when triggered.Users also asked about the temperature on that particular time as well as future forecasts, attimes asking for a specific day, for example, ?Alexa, is it gonna snow two days from now?? \cite{c6}


%%%
This is a source \cite{c8} Deployed within a piece of equipment (smartphone, smart speaker, vehicle), the voice assistant is on standby. In concrete terms, it is constantly listening, while possibly concentrating its listening on certain areas of space, for example to try to neutralise sound sources such as a television set (using the spatial filtering techniques such as beam forming). However, it does not store the audio data and does not perform any operations until a specific wake word has been heard. For this purpose, a buffer of a few seconds is used. \cite{c8} When the user utters the wake word, the assistant "wakes  up".  A  listening  channel  opens  and  the  audio  content is streamed. In many cases, if the processing is done remotely, a second check of the wake word pronunciation is per-formed  on  the  server  side  in  order  to  limit  unwanted  triggering. The user shall state his/her request, which is forwarded to the processing entities.  Using natural language processing (NLP) technology, speech is interpreted. Automatic speech recognition used to involve three distinct steps to: 1) determine which phonemes had been pronounced using an acoustic model; 2) determine which words were pronounced using a phonetic dictionary; 3) transcribe the sequence of words (sentence) most likely to have been spoken using a language model. Today, with the progress made possible by deep learning (a machine learning technique), a large number of systems offer automatic transcription of speech from end to end. 




\section{PROCEDURE FOR PAPER SUBMISSION}

\subsection{Figures and Tables}

Positioning Figures and Tables: Place figures and tables at the top and bottom of columns.
   \begin{figure}[thpb]
      \centering
      \framebox{\parbox{3in}{We suggest that you use a text box to insert a graphic (which is ideally a 300 dpi TIFF or EPS file, with all fonts embedded) because, in an document, this method is somewhat more stable than directly inserting a picture.
}}
      %\includegraphics[scale=1.0]{figurefile}
      \caption{Inductance of oscillation winding on amorphous
       magnetic core versus DC bias magnetic field}
      \label{figurelabel}
   \end{figure}
   


%%%%%%%%%%%%%%%%%%%



\section*{APPENDIX}

Appendixes should appear before the acknowledgment.

\section*{ACKNOWLEDGMENT}

The preferred spelling of the word �acknowledgment� in America is without an �e� 


%%%%%%%%%%%%%%%%%%%%%%%%%%%%%%%%%%%%%%%%%%%%%%%%%%%%%%%%%%%%%%%%%%%%%%%%%%%%%%%%

References are important to the reader; therefore, each citation must be complete and correct. If at all possible, references should be commonly available publications.
bla

%%%%%%%%%%%%

\addtolength{\textheight}{-12cm}   % This command serves to balance the column lengths
% on the last page of the document manually. It shortens
% the textheight of the last page by a suitable amount.
% This command does not take effect until the next page
% so it should come on the page before the last. Make
% sure that you do not shorten the textheight too much.


%%%%%%%%%%%%%%%%%%%%%%%%%%%%%%%%%%%%%%%%%%%%%%%%%%%%%%%%%%%%%%%%%%%%%%%%%%%%%%%%

\begin{thebibliography}{99}

\bibitem{c1} Nan Z., Xianghang M., Xuan F. Dangerous Skills: Understanding and Mitigating Security Risks of Voice-Controlled Third-Party Functions on Virtual Personal Assistant Systems. \url{https://wiki.aalto.fi/download/attachments/116657996/IoT-attestation.pdf}. date accessed: 29.04.2021

\bibitem{c2} Terzopoulos G., Satratzemi M.. Voice Assistants and Smart Speakers in Everyday Life and in Education. \url{https://files.eric.ed.gov/fulltext/EJ1267812.pdf}. date accessed: 01.05.2021

\bibitem{c3} Druga, S., Williams, R., Breazeal, C., Resnick, M. Hey Google is it OK if I eat you?: Initial explorations in child-agent interaction. \url{https://dl.acm.org/doi/pdf/10.1145/3078072.3084330}. date accessed: 02.05.2021

\bibitem{c4}Malkin, N., Deatrick, J., Tong, A., Wijesekera, P., Egelman, S., Wagner, D. Privacy Attitudes of Smart Speaker Users.  \url{https://www.researchgate.net/publication/336184996_Privacy_Attitudes_of_Smart_Speaker_Users}. date accessed: 02.05.2021

\bibitem{c5} Lau, J., Zimmerman, B., Schaub, F. Alexa, are you listening?: Privacy perceptions, concerns and privacy-seeking behaviors with smart speakers. \url{https://www.key4biz.it/wp-content/uploads/2018/11/cscw102-lau-1.pdf}. date accessed: 02.05.2021

\bibitem{c5a} Lau, J., Zimmerman, B., Schaub, F. Alexa, Stop Recording! : Mismatches between Smart Speaker Privacy Controls and User Needs. \url{https://www.usenix.org/sites/default/files/soups2018posters-lau.pdf}. date accessed: 03.05.2021

\bibitem{c6} Ammari T., Kaye J., Tsai J.T., Bentley F. Music, search and IoT: How people (really) use voice assistants. \url{https://www.researchgate.net/publication/332745214_Music_Search_and_IoT_How_People_Really_Use_Voice_Assistants}. date accessed: 03.05.2021

\bibitem{c7} Amazon.com. Alexa features. \url{https://www.amazon.com/b?ie=UTF8&node=21576558011}. date accessed: 03.05.2021

\bibitem{c8} Commission Nationale de l'Informatique et des Libert�s. White Paper Collection. Exploring the ethical, technical and legal issues of voice assistants. \url{https://www.cnil.fr/sites/default/files/atoms/files/cnil_white-paper-on_the_record.pdf}. date accessed: 27.04.2021

\bibitem{c9} Lei X., Tu L.,Liu A.X., Chi-Yu Li. The Insecurity of Home Digital Voice Assistants: Vulnerabilities, Attacks and Countermeasures. \url{https://ieeexplore.ieee.org/abstract/document/8433167}. date accessed: 03.05.2021

\bibitem{c10} Ren J., Dubois D. J., Choffnes D.,  Mandalar A. M. Information Exposure From Consumer IoT Devices. \url{https://www.researchgate.net/publication/336657694_Information_Exposure_From_Consumer_IoT_Devices_A_Multidimensional_Network-Informed_Measurement_Approach}. date accessed: 03.05.2021

\bibitem{c11} Knote R., Janson A., Eigenbrod L., S�llner M.. The What and How of Smart Personal Assistants: Principles and Application Domainsfor IS Research.  \url{https://mkwi2018.leuphana.de/wp-content/uploads/MKWI_285.pdf}. date accessed: 03.05.2021


\end{thebibliography}




\end{document}
