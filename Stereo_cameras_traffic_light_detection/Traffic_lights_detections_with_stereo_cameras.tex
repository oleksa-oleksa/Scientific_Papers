% !TeX spellcheck = en_GB 
%%%%%%%%%%%%%%%%%%%%%%%%%%%%%%%%%%%%%%%%%%%%%%%%%%%%%%%%%%%%%%%%%%%%%%%%%%%%%%%%
%2345678901234567890123456789012345678901234567890123456789012345678901234567890
%        1         2         3         4         5         6         7         8

\documentclass[letterpaper, 10 pt, conference]{ieeeconf}  % Comment this line out if you need a4paper

%\documentclass[a4paper, 10pt, conference]{ieeeconf}      % Use this line for a4 paper

\IEEEoverridecommandlockouts                              % This command is only needed if 
                                                          % you want to use the \thanks command

\overrideIEEEmargins                                      % Needed to meet printer requirements.

% See the \addtolength command later in the file to balance the column lengths
% on the last page of the document

% The following packages can be found on http:\\www.ctan.org
%\usepackage{graphics} % for pdf, bitmapped graphics files
%\usepackage{epsfig} % for postscript graphics files
%\usepackage{mathptmx} % assumes new font selection scheme installed
%\usepackage{times} % assumes new font selection scheme installed
%\usepackage{amsmath} % assumes amsmath package installed
%\usepackage{amssymb}  % assumes amsmath package installed
\usepackage{url}
\usepackage{graphicx}
\usepackage{hyperref}



\title{\LARGE \bf
Stereo cameras with applications to traffic scenarios,\\ traffic lights detection
}


\author{\textbf{Oleksandra Baga}\\ 
\textit{\small Master Computer Science, Freie Universit\"at Berlin}\\ 
\textit{\small Sommersemester 2021, Seminar KI / Autonomes Fahren}\\
{\small oleksandra.baga@gmail.com}}

\begin{document}


\maketitle

\thispagestyle{empty}
\pagestyle{empty}


%%%%%%%%%%%%%%%%%%%%%%%%%%%%%%%%%%%%%%%%%%%%%%%%%%%%%%%%%%%%%%%%%%%%%%%%%%%%%%%%
\begin{abstract}

A traffic light recognition system is a very important building block in an advanced driving  assistance  system  and  an  autonomous  vehicle  system.

\end{abstract}


%% STRUCTURE
\section{INTRODUCTION}
A stereo camera is a type of camera with two or more image sensors. This allows the camera to simulate human binocular vision and therefore gives it the ability to perceive depth.

%%%%%%%%%%%

\section{ABOUT}
\subsection{What is about?}


TODO

Lorem Ipsum

\subsection{Something}

TODO

Sapana navaga

\section{CONCLUSIONS}
TADAM! 

THE END


\addtolength{\textheight}{-12cm}   % This command serves to balance the column lengths
% on the last page of the document manually. It shortens
% the textheight of the last page by a suitable amount.
% This command does not take effect until the next page
% so it should come on the page before the last. Make
% sure that you do not shorten the textheight too much.


%%%%%%%%%%%%%%%%%%%%%%%%%%%%%%%%%%%%%%%%%%%%%%%%%%%%%%%%%%%%%%%%%%%%%%%%%%%%%%%%

\begin{thebibliography}{99}

\bibitem{c1} Andreas Fregin, Julian M\"uller, Klaus Dietmayer. Three  Ways  of  using  Stereo  Vision  for  Traffic  Light  Recognition \url{https://www.researchgate.net/publication/318810474} date accessed: 15.05.2021

\bibitem{c2} Andreas Fregin, Julian M\"uller, Klaus Dietmayer. Multi-camera system for traffic light detection: About camera setup and mapping of detections \url{https://www.researchgate.net/publication/323792381} date accessed: 19.05.2021

\bibitem{c3} iroki Moizumi, Yoshihiro Sugaya, Masako Omachi, Shinichiro Omachi. Traffic Light Detection Considering Color Saturation Using In-Vehicle Stereo Camera \url{https://doi.org/10.2197/ipsjjip.24.349} date accessed: 19.05.2021

\end{thebibliography}
\end{document}
