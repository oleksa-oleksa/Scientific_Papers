% !TeX spellcheck = en_GB 
%%%%%%%%%%%%%%%%%%%%%%%%%%%%%%%%%%%%%%%%%%%%%%%%%%%%%%%%%%%%%%%%%%%%%%%%%%%%%%%%
%2345678901234567890123456789012345678901234567890123456789012345678901234567890
%        1         2         3         4         5         6         7         8

\documentclass[letterpaper, 10 pt, conference]{ieeeconf}  % Comment this line out if you need a4paper

%\documentclass[a4paper, 10pt, conference]{ieeeconf}      % Use this line for a4 paper

\IEEEoverridecommandlockouts                              % This command is only needed if 
                                                          % you want to use the \thanks command

\overrideIEEEmargins                                      % Needed to meet printer requirements.

% See the \addtolength command later in the file to balance the column lengths
% on the last page of the document

% The following packages can be found on http:\\www.ctan.org
%\usepackage{graphics} % for pdf, bitmapped graphics files
%\usepackage{epsfig} % for postscript graphics files
%\usepackage{mathptmx} % assumes new font selection scheme installed
%\usepackage{times} % assumes new font selection scheme installed
%\usepackage{amsmath} % assumes amsmath package installed
%\usepackage{amssymb}  % assumes amsmath package installed
\usepackage{url}
\usepackage{graphicx}
\usepackage{hyperref}



\title{\LARGE \bf
Home Digital Voice Assistants: use cases and vulnerabilities
}


\author{\textbf{Oleksandra Baga}\\ 
\textit{\small Master Computer Science, Freie Universit\"at Berlin}\\ 
\textit{\small Sommersemester 2021, Seminar Technische Informatik}\\
{\small oleksandra.baga@gmail.com}}

\begin{document}


\maketitle

\thispagestyle{empty}
\pagestyle{empty}


%%%%%%%%%%%%%%%%%%%%%%%%%%%%%%%%%%%%%%%%%%%%%%%%%%%%%%%%%%%%%%%%%%%%%%%%%%%%%%%%
\begin{abstract}

Smart speakers with voice assistants achieved last years impressive results in speech recognition enabling more seamless interactions between user and a machine but also raise privacy concerns due to their continuously listening microphones. A better understanding of these aspects can help future smart speaker users to make a right decision about the digitalisation of their homes. For these purposes this paper contains as well a result of research about the functionality of digital voice assistance and the real use cases how people are tending to use a device as the research of actual security and privacy concerns including attack surfaces and vulnerabilities.  
\end{abstract}


%% STRUCTURE
\section{INTRODUCTION}
The development of the Deep Learning algorithms and Internet of Things last years is opening up a new era in the use of the digital tools that surround us. A combination of various algorithms in Machine Learning, Deep Learning, speech synthesis, and Natural Language Processing (NLP) to providing services to the users makes it possible to achieve impressive results in speech recognition enabling more seamless interactions between user and a machine. It is realistic now to say that in the coming years digital voice assistances probably will come into the use of every household. They  could  become  embedded  in  users'  day-to-day  routines,  particularly  people  in  a  dependent situation, whether elderly or disabled. 

However, these undeniable advances should not obscure the questions that voice assistants raise from a data protection  perspective,  in  particular  from  the  point  of  view of transparency in the way their system functions \cite{c8}. In the survey by Lau et al. \cite{c5} smart speaker users and non-users were interviewed to find out their arguments for and against adopting this new technology and their privacy perceptions and concerns. Many non-users believe that these devices are not useful at all and companies are not to be trusted. On the other hand, smart speaker users have fewer privacy concerns and rely on companies to safeguard their personal data which think are not interesting to others \cite{c2}. The goal of this research is reality check of privacy concerns and myths circulating about voice assistants and the abilities that they are assumed to have. This paper presents the closer look at digital voice assistance functions for a clearer understanding of the logic behind these systems and the security questions they raise for their users. 

%%%%%%%%%%%

\section{ABOUT}
\subsection{What is about?}


TODO

Lorem Ipsum

\subsection{Something}

TODO

Sapana navaga

\section{CONCLUSIONS}
TADAM! 

THE END


\addtolength{\textheight}{-12cm}   % This command serves to balance the column lengths
% on the last page of the document manually. It shortens
% the textheight of the last page by a suitable amount.
% This command does not take effect until the next page
% so it should come on the page before the last. Make
% sure that you do not shorten the textheight too much.


%%%%%%%%%%%%%%%%%%%%%%%%%%%%%%%%%%%%%%%%%%%%%%%%%%%%%%%%%%%%%%%%%%%%%%%%%%%%%%%%

\begin{thebibliography}{99}

\bibitem{c1} Nan Z., Xianghang M., Xuan F. Dangerous Skills: Understanding and Mitigating Security Risks of Voice-Controlled Third-Party Functions on Virtual Personal Assistant Systems. \url{https://wiki.aalto.fi/download/attachments/116657996/IoT-attestation.pdf}. date accessed: 29.04.2021


\end{thebibliography}
\end{document}
